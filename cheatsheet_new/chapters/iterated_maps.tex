\section*{Iterated Maps}

\textbf{Discrete-time dynamical systems} have a state which is only 
defined at integer steps. 

The state is:
$\xx(n) = (x_1(n), \dots, x_k(n))$
where $k$ is the dimension of the system and $n$ is an integer step parameter

and the rule is: $\xx_n = \mathbf{f}(\xx_{n-1}, \dots, \xx_{n-m})$

Next state is obtained by directly apply the map $\mathbf{f}$

\section*{1D Iterated Maps and Cobweb Plots}

We focus on the simplest case: a 1D map $x_n = f(x_{n-1})$

\begin{tightitemize}
    \item \textbf{Orbit:} sequence of points generated starting at $x_0$
    and keep applying the map 
    \item \textbf{Fixed point:} Point that maps to itself: $f(x^*) = x^*$
\end{tightitemize}

\textbf{Sawtooth diagram:} Plotting the $(x, f(x))$ diagram. No time shown, 
all it tells you is if the system is at $x$ now, the next state will be at $f(x)$

\textbf{Cobweb Plot construction algorithm:}

\begin{tightitemize}
    \item Draw $y=f(x)$ and $y=x$
    \item Start at initial point $x_0$ on horizontal axis 
    \item Move \textbf{vertically} to $y = f(x)$ (this is point $x_0, x_1$)
    \item Move \textbf{horizontally} to $y = x$ (this is point $x_1, x_1$)
    \item Move \textbf{vertically} to $y = f(x)$ again (this is point $x_1, x_2$)
    \item Keep going lil bro
\end{tightitemize}

The intersections of $y = f(x)$ and $y = x$ are fixed points. We can analyze 
stability of a fixed point by looking at the plot.

\section*{Stability of fixed points}

A linear approximation shows that $\epsilon_{n+1} = f'(x^*)\epsilon_n$. Let $\lambda = f'(x^*)$

Solving: $\epsilon_n = \lambda^n \epsilon_0$. If you know some basics you can 
infer stability from this alone but I will write a table.

\begin{tightitemize}
  \item \textbf{Stable (Attracting)}
  \begin{itemize}
    \item $0 < \lambda < 1$: Monotonic — Converges from one side
    \item $-1 < \lambda < 0$: Oscillatory — Zig-zag convergence (alternating sides)
    \item $\lambda = 0$: Superstable — Very fast convergence
  \end{itemize}

  \item \textbf{Unstable (Repelling)}
  \begin{itemize}
    \item $\lambda > 1$: Monotonic — Diverges from one side
    \item $\lambda < -1$: Oscillatory — Alternating divergence
  \end{itemize}

  \item \textbf{Marginal}
  \begin{itemize}
    \item $|\lambda| = 1$: Neutral — Linearization inconclusive
  \end{itemize}
\end{tightitemize}

\textbf{Logistic Map:} $x_{n+1} = r x_n (1 - x_n)$ where $x_n \in [0, 1]$ is 
population, $r \in [0, 4]$ is growth rate

\begin{tightitemize}
    \item $1<r<3$: The population converges to a single, stable fixed point $x^*=1-1/r$.
    \item $r=3$: The fixed point becomes unstable.
    \item $3<r<3.449\dots$: The system no longer settles to one point. It settles into a stable period-2 cycle, oscillating between two values. This is a bifurcation.
    \item $r>3.449\dots$: The 2-cycle becomes unstable and splits into a stable period-4 cycle. This continues, creating an 8-cycle, 16-cycle, etc., in a period-doubling cascade.
    \item $r>r^{\infty} \approx 3.5699\dots$: The cascade finishes, and the system enters the chaotic regime. The orbit becomes aperiodic, never settling down and seemingly random.
\end{tightitemize}

This behavior can be summarized in an \textbf{orbit diagram}.
\begin{tightitemize}
    \item The x-axis is the parameter r.
    \item The y-axis plots the long-term attractor points for that r.
\end{tightitemize}    

A \textbf{bifurcation} is a qualitative change in the long-term behavior
of a system with a smooth variation of a parameter
    
$\lambda = \lim_{n \to \infty} \frac{1}{n} \sum_{k=0}^{n-1} \ln|f'(x_k)|$

The approximation for $\lambda$ can be constructed numerically, by iterating the map!
\begin{tightitemize}
    \item $\lambda < 0$: For stable fixed points and cycles
    \item $\lambda > 0$: For chaotic attractors
    \item $\lambda = 0$: This is the marginal case, which occurs at bifurcation points.
    \item $\lambda$ is the same for all points in the basin of attraction of an attractor
\end{tightitemize} 
    

