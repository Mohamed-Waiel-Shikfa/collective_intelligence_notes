\section*{Cellular Automata (CA)}

\section*{CA properties}

A \textbf{Cellular Automaton} is a multi-dimensional discrete-time dynamical
system that is defined by
the principle of locality. properties:

\begin{description}
    \item[Systems's state is $n$-dimensional] $\xx(t) = (x_1(t), \dots, x_n(t))$
    \item[Discrete time]: updated at discrete time steps
    \item[State components arranged according to a given topology]
    \item[Neighborhood defined based on topology]: $N(x_i) =
    {x_j : x_j \text{ is a neighbor of }x_i}$

    Neighborhood is the range for a cell to be influenced by other cells

    1D Example: $N(x_i)={x_{i-1},x_i,x_{i+1}}$

    2D Examples:
    \begin{tightitemize}
        \item Von Neumann: The cell and its 4 neighbors (up, down, left, right).
        \item Moore: The cell and its 8 surrounding neighbors (a 3x3 box).
    \end{tightitemize}

    \item[Locality of updates]: Each state component $x_i$ evolves according to a rule
    that depends only on its own state and those of its neighbors in $N(x_i)$.

    Local-state transition function, $F_i: S(N(x_i)) \to S_i$ where $S_i$
    s the set of values that state component $x_i$ can take,
    and $S$ is the state values from the cells in the neighborhood set.
\end{description}

\section*{Lattices and Boundaries:}
\textbf{Infinite/adaptive lattice}: The grid grows as the pattern propagates

\textbf{Finite lattice}
\begin{tightitemize}
    \item Hard boundary: fixed, edge cells have a fixed state
    \item Hard boundary: reflective, leftmost (rightmost) cell only diffuse right (left)
    \item Soft boundary: periodic boundary conditions, edges wrap around
\end{tightitemize}

\section*{Updating Schedules:} Synchronous, Asynchronous

\section*{Some Math}
We assume they are homogeneous (same lattice, same N, and same rule F for all cells)
and use synchronous updating.

\textbf{Combinatorics}
Let:
$k = |S|$ is the number of states per cell

$M$ is the number of cells

$r$ is the range ($floor(|N(a)|/2)$)

Then:
Number of possible state configs: $k^{M}$
,Number of possible neighborhood configurations: $k^{|N|}$
, Number of possible evolution functions (rules): $k^{\left(k^{|N|}\right)}$


\section*{Wolfram Code}

For Wolfram's 1D CA: $S = \{0, 1\}$ (so $k=2$) and the neighborhood is $r=1$ (i.e., $N(x_i) = \{x_{i-1}, x_i, x_{i+1}\}$, so $|N|=3$).

Using our formula, the number of rules = $k^{\left(k^{|N|}\right)} = 2^{\left(2^3\right)} = 2^8 = 256$.


\section*{Four classes of Behavior and Chaos}

\renewcommand{\arraystretch}{1.3}
\begin{table}[h!]
\centering
\begin{tabular}{|c|p{4cm}|p{4cm}|c|}
\hline
\textbf{Class} & \textbf{Behavior} & \textbf{Information Dynamics} & \textbf{Lyapunov Exponent} \\
\hline
1 & Evolves to a simple, stable, homogeneous state (all 0s or all 1s). & Small changes die; information is lost. & $\lambda \le 0$ \\
\hline
2 & Evolves to simple periodic structures (stripes, oscillators). & Small changes may persist locally but do not spread. & $\lambda = 0$ \\
\hline
3 & Evolves to chaotic, aperiodic patterns (e.g., Rule 30). & Small changes spread out and affect distant regions. & $\lambda > 0$ \\
\hline
4 & Creates complex, localized, moving structures. (e.g., Rule 110 is \textbf{Turing complete}) & Small changes may or may not spread; irregular but structured dynamics. & $\lambda > 0$ (tends to 0) \\
\hline
\end{tabular}
\end{table}

\section*{Game of Life}

\begin{tightitemize}
    \item 2D Lattice of identical cells
    \item Moore neighborhood
    \item 2 states: dead or alive
    \item The 4 Rules:
    \begin{description}
        \item[Loneliness] A live cell with $<2$ live neighbors dies.
        \item[Overcrowding] A live cell with $>3$ live neighbors dies.
        \item[Survival] A live cell with 2 or 3 live neighbors lives on.
        \item[Reproduction] A dead cell with exactly 3 live neighbors becomes alive.
    \end{description}
    \item Garden of Eden: A pattern that can only exist as initial pattern. In other
words, no parent could possibly produce the pattern.
\end{tightitemize}
